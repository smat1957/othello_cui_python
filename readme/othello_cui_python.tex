% latex uft-8
\documentclass[uplatex,a4paper,11pt,oneside,openany]{jsbook}
%
\usepackage[dvipdfmx]{graphicx}
\usepackage{amsmath,amssymb}
\usepackage{bm}
\usepackage{graphicx}
\usepackage{ascmac}
\usepackage{setspace}
\usepackage{here}
\usepackage{listings,jlisting} %日本語のコメントアウトをする場合jlistingが必要
%ここからソースコードの表示に関する設定
\usepackage{xcolor,comment}

\definecolor{mygreen}{rgb}{0,0.6,0}
\definecolor{mygray}{rgb}{0.5,0.5,0.5}
\definecolor{mymauve}{rgb}{0.58,0,0.82}

\begin{comment}
\lstset{
  backgroundcolor=\color{white},   % choose the background color; you must add \usepackage{color} or \usepackage{xcolor}; should come as last argument
  basicstyle=\footnotesize,        % the size of the fonts that are used for the code
  breakatwhitespace=false,         % sets if automatic breaks should only happen at whitespace
  breaklines=true,                 % sets automatic line breaking
  captionpos=b,                    % sets the caption-position to bottom
  commentstyle=\color{mygreen},    % comment style
  deletekeywords={...},            % if you want to delete keywords from the given language
  escapeinside={\%*}{*)},          % if you want to add LaTeX within your code
  extendedchars=true,              % lets you use non-ASCII characters; for 8-bits encodings only, does not work with UTF-8
  firstnumber=1000,                % start line enumeration with line 1000
  frame=single,	                   % adds a frame around the code
  keepspaces=true,                 % keeps spaces in text, useful for keeping indentation of code (possibly needs columns=flexible)
  keywordstyle=\color{blue},       % keyword style
  language=Octave,                 % the language of the code
  morekeywords={*,...},            % if you want to add more keywords to the set
  numbers=left,                    % where to put the line-numbers; possible values are (none, left, right)
  numbersep=5pt,                   % how far the line-numbers are from the code
  numberstyle=\tiny\color{mygray}, % the style that is used for the line-numbers
  rulecolor=\color{black},         % if not set, the frame-color may be changed on line-breaks within not-black text (e.g. comments (green here))
  showspaces=false,                % show spaces everywhere adding particular underscores; it overrides 'showstringspaces'
  showstringspaces=false,          % underline spaces within strings only
  showtabs=false,                  % show tabs within strings adding particular underscores
  stepnumber=2,                    % the step between two line-numbers. If it's 1, each line will be numbered
  stringstyle=\color{mymauve},     % string literal style
  tabsize=2,	                   % sets default tabsize to 2 spaces
  title=\lstname                   % show the filename of files included with \lstinputlisting; also try caption instead of title
}
\end{comment}

%ここからソースコードの表示に関する設定

\lstdefinestyle{customc}{
  belowcaptionskip=1\baselineskip,
  breaklines=true,
  numbers=left,
  frame=single,
  xleftmargin=\parindent,
  language=C,
  showstringspaces=false,
  basicstyle=\footnotesize\ttfamily,
  keywordstyle=\bfseries\color{green!40!black},
  commentstyle=\itshape\color{purple!40!black},
  identifierstyle=\color{blue},
  stringstyle=\color{orange},
}

\lstdefinestyle{custompython}{
  belowcaptionskip=1\baselineskip,
  breaklines=true,
  numbers=left,
  frame=single,
  xleftmargin=\parindent,
  language=Python,
  showstringspaces=false,
  basicstyle=\footnotesize\ttfamily,
  keywordstyle=\bfseries\color{green!40!black},
  commentstyle=\itshape\color{purple!40!black},
  identifierstyle=\color{blue},
  stringstyle=\color{orange},
}

\lstdefinestyle{customjava}{
  belowcaptionskip=1\baselineskip,
  breaklines=true,
  numbers=left,
  frame=single,
  xleftmargin=\parindent,
  language=java,
  showstringspaces=false,
  basicstyle=\footnotesize\ttfamily,
  keywordstyle=\bfseries\color{green!40!black},
  commentstyle=\itshape\color{purple!40!black},
  identifierstyle=\color{blue},
  stringstyle=\color{orange},
}

\lstdefinestyle{customasm}{
  belowcaptionskip=1\baselineskip,
  frame=L,
  xleftmargin=\parindent,
  language=[x86masm]Assembler,
  basicstyle=\footnotesize\ttfamily,
  commentstyle=\itshape\color{purple!40!black},
}

\lstset{escapechar=@,style=custompython}

\makeatletter
\def\ps@plainfoot{%
  \let\@mkboth\@gobbletwo
  \let\@oddhead\@empty
  \def\@oddfoot{\normalfont\hfil-- \thepage\ --\hfil}%
  \let\@evenhead\@empty
  \let\@evenfoot\@oddfoot}
  \let\ps@plain\ps@plainfoot
\renewcommand{\chapter}{%
  \if@openright\cleardoublepage\else\clearpage\fi
  \global\@topnum\z@
  \secdef\@chapter\@schapter}
\makeatother
%
\newcommand{\maru}[1]{{\ooalign{%
\hfil\hbox{$\bigcirc$}\hfil\crcr%
\hfil\hbox{#1}\hfil}}}
%
\setlength{\textwidth}{\fullwidth}
\setlength{\textheight}{40\baselineskip}
\addtolength{\textheight}{\topskip}
\setlength{\voffset}{-0.55in}
%
\begin{document}
% START DOCUMENT
%
% COVER
\begin{center}
  \huge \par
  \vspace{15mm}
  \huge \par
  \vspace{15mm}
  \LARGE Othello (CUI - Python) \par
  \vspace{100mm}
  \Large \date \par
  \vspace{15mm}
  \Large \par
  \vspace{10mm}
  \Large S.Matoike \par
  \vspace{10mm}
\end{center}
\thispagestyle{empty}
\clearpage
\addtocounter{page}{-1}
\newpage
\setcounter{tocdepth}{3}
%
\tableofcontents
%

\chapter{Othello(リバーシ):N $\times$ N}

8 $\times$ 8サイズのゲームが一般的ですが、
6 $\times$ 6盤の縮小オセロでは必勝法が見つかっているようです

1993年にイギリスのJoel Feinsteinが、6 $\times$ 6盤の縮小オセロで、
先手が最善手を尽くしたとしても後手が「20対16」で勝つ、
(先手16対、後手20で、後手の4目勝ち)という事を示したということです

ここでは、
4 $\times$ 4「以上」の大きさの盤面に対応するオセロを作ります。
あなたにゲームを最後まで続ける根気があるなら、
10 $\times$ 10でも20 $\times$ 20でもゲームすることができるかもしれません

\section{CUI版}

以下のゲーム例では、×(BLACK)の手番ですが、○(WHITE)を裏返せないところに×を置こうとしたり、
×(BLACK)を置ける場所があるのにpassを宣言したりすると、再入力を促されています

○(WHITE)の手番で、まずは乱数によってコンピュータが石を置くスロットを決めていますが、
後ほどStrategyクラスで、コンピュータの強い戦略について考察していきます

\begin{spacing}{0.74}
\begin{verbatim}
  WHITE:2, BLACK:2
    0 1 2 3 →x
  0 . . . .
  1 . X O .
  2 . O X .
  3 . . . .
  ↓
  y
  Your turn. [yx] or "pass": 23
  Your turn. [yx] or "pass": pass
  Your turn. [yx] or "pass": 13

  WHITE:1, BLACK:4
    0 1 2 3 →x
  0 . . . .
  1 . X X X
  2 . O X .
  3 . . . .
  ↓
  y
  WHITE:3, BLACK:3
    0 1 2 3 →x
  0 . . . .
  1 . X X X
  2 . O O O
  3 . . . .
  ↓
  y
  Your turn. [yx] or "pass": 33

  WHITE:1, BLACK:6
    0 1 2 3 →x
  0 . . . .
  1 . X X X
  2 . O X X
  3 . . . X
  ↓
  y
  WHITE:3, BLACK:5
    0 1 2 3 →x
  0 . O . .
  1 . O X X
  2 . O X X
  3 . . . X
  ↓
  y
  Your turn. [yx] or "pass": 00

  ( ・・ 途中略 ・・ )

  WHITE:5, BLACK:9
    0 1 2 3 →x
  0 X X X O
  1 O X X X
  2 O O X X
  3 O . . X
  ↓
  y
  Your turn. [yx] or "pass": 31

  WHITE:4, BLACK:11
    0 1 2 3 →x
  0 X X X O
  1 O X X X
  2 O X X X
  3 O X . X
  ↓
  y
  WHITE:7, BLACK:9
    0 1 2 3 →x
  0 X X X O
  1 O X X X
  2 O O X X
  3 O O O X
  ↓
  y

  Winner : BLACK
\end{verbatim}
\end{spacing}

%\section{Human vs. Machine}

\subsection{Gameクラス}

メインでは、Gameクラスのコンストラクタでへの引数で盤面のサイズ(N=4以上の偶数)、
先手を'human', 'machine'から、戦略を'random', 'maxflip', 'minimax', 'alphabeta'から指定します

\begin{lstlisting}[caption=main,label=othello00]
from Game import Game

if __name__ == '__main__':
    sente = 'human'         # 'human, 'machine'
    senryaku = 'maxflip'    # 'random', 'maxflip', 'minimax', 'alphabeta'
    game = Game( N=6, turn=sente, strategy=senryaku )      # board size -> N x N
    game.start()
\end{lstlisting}

Gameクラスのコンストラクタで、盤面クラス(Board)のインスタンス(board)、
人クラス(Human)のインスタンス(human)、
コンピュータクラス(Machine)のインスタンス(machine)を生成し、
インスタンス変数self.playerには、humanとmachineの2つのインスタンスを要素とするリストを用意しています

start()メソッドで、ゲームの進行の主要部分を記述しています

①盤面を印刷(board.print())し、②プレーヤの打つ手を選び(player[turn].Te(self.board))、
③もし、パス(Board.PASS)でなかった場合は、④盤面に石を置く(board.putStone(te,color))
ことになります。
⑤パスならば盤面に石を置かずに、プレーヤを交代(turn = (turn+1)\%2)します

⑥盤面の状況から勝敗の判定(board.Winner())を行い、戻り値が
Board.GAMEならばゲームを継続、それ以外
(BLACK\_WINかWHITE\_WINかDRAW)ならば、ゲームを終了して結果の勝敗を表示します

\begin{lstlisting}[caption=Game class,label=othello01]
from Board import Board
from Human import Human
from Machine import Machine
from Stone import Stone

class Game():
    def __init__(self, N=4, turn='human', strategy='random'):
        self.board = Board(NW=N)
        human = Human(color=Stone.BLACK)
        machine = Machine(color=Stone.WHITE, strategy=strategy)
        self.player = [human, machine]
        self.turn = 0
        if turn=='machine':
            self.turn = 1

    def start(self):
        turn = self.turn
        winner = Board.GAME
        while winner==Board.GAME:
            self.board.print()                    # ①

            te = self.player[turn].Te(self.board) # ②
            if te!=Board.PASS:                    # ③
                color = self.player[turn].color
                self.board.put_stone(te,color)    # ④

            turn = (turn+1)%2                     # ⑤
            winner = self.board.winner()          # ⑥

        self.board.print()
        print()
        if winner==Board.BLACK_WIN:
            print('Winner : BLACK')
        elif winner==Board.WHITE_WIN:
            print('Winner : WHITE')
        elif winner==Board.DRAW:
            print('DRAW')
\end{lstlisting}

\subsection{Stoneクラス}

石のクラスでは、BLACK、WHITE、EMPTYの中の何れかの色特性(self.color)、
及び盤面上の位置(self.locate)をプロパティとして持たせています

用意しているメソッドは、各プロパティに関するセッターやゲッターです

\begin{lstlisting}[caption=Stone class,label=othello04]
class Stone():
    EMPTY = 0
    BLACK = 1
    WHITE = 3 - BLACK
    MACHINE_ID = WHITE
    HUMAN_ID = BLACK
    def __init__(self, te, color):
        self.color = color
        self.locate = te

    def getLocate(self):
        return self.locate

    def setColor(self, color):
        self.color = color

    def getColor(self):
        return self.color

    def flipColor(self):
        self.color = 3 - self.color

    def getFigure(self):
        if self.color==Stone.BLACK:
            return 'X'
        elif  self.color==Stone.WHITE:
            return 'O'
        return '.'
\end{lstlisting}

\subsection{Boardクラス}

このクラスでは、盤面の状態をリスト(self.stones)に保持しています

コンストラクタでは、
まず引数で示されたサイズ(NW * NW)の空の(Stone.EMPTY)盤面を用意しています

次に、盤面の中央にBLACKとWHITEの石をそれぞれ2個ずつ配置しますが、
配置のパタンは乱数によって決めています

石の数を nWhite,nBlack,nEmpty に保持させ、
winner()メソッドの最初でそれぞれの値を数えています

nWhite+nBlack+nEmptyは盤面のスロットの総数になりますから、
nEmptyがゼロになった段階で、nWhiteとnBlackの大小関係から
勝敗を決定しています
(nEmptyがゼロになるより前に勝敗の判定がつく場合について、
具体的にプログラムする必要があるかもしれません)

print()メソッドは、画面に盤面の状態を表示しています

is\_empty()メソッドは、引数で指定したスロットが空(Stone.EMPTY)
であるか否かを判定しています

empty\_list()メソッドは、現在の盤面において、
空(Stone.EMPTY)のスロット番号の一覧をリストにして返します

can\_put(te, color)メソッドは、引数teで示されたスロット番号の場所に、
第2引数のcolorで示された色の石を置けるかどうかを判定しています
(具体的には、self.check\_flip( Stone(te, color) )を呼び出して、
色を反転できる石が何個かあるなら、そこには石を置けるという判断をしています)

check\_flip(stone)メソッドでは、
引数で受け取っているstoneオブジェクトは、その石を置こうとしているスロットの位置番号、
及びその石の色をプロパティとして持っていますから、盤面のその位置に指定された色の
石を置いたときに、上下、左右、右斜め上、右斜め下、左斜め上、左斜め下の8つの方向で
何個の石を反転できるかを数えています(最後に、反転できる石の総数を返しています)

flip\_list(stone)メソッドでは、
check\_flip(stone)メソッドで数えた各方向での反転できる石の数に基づいて、
反転する石のスロット番号のリストを作って返しています

put\_stone(te, color)メソッドは、
まずは、指定されたスロット位置に指定の石を置いた後に、
反転する石のリストをflip\_list(stone)メソッドで作って、
そのリストをset\_stones(flist,color)に渡して、
実際に盤面の石を反転させるようにします

set\_stones(flist,color)メソッドは、受け取ったリストを基に
盤面のデータを更新し、石を反転させています

reset\_stones()メソッドは、
set\_stones()メソッドで反転させた盤面上の石を元に戻します

puttable()メソッドは、can\_put()メソッドがTrueだった場合のnKomaリストを取得します

puttable\_list()メソッドは、puttbale()メソッドの戻り値から、
盤面上で石を置ける場所のリストを取得します

\begin{lstlisting}[caption=Board class,label=othello05]
from random import randint
from Stone import Stone

class Board:
    PASS = -10
    GAME =  -1
    DRAW =   0
    HUMAN_WIN = BLACK_WIN = Stone.BLACK * 100
    MACHINE_WIN = WHITE_WIN = Stone.WHITE * 100
    UNITV = ((0,-1),(1,-1),(1,0),(1,1),(0,1),(-1,1),(-1,0),(-1,-1))

    def __init__(self, NW=4):
        self.nKoma = [0 for _ in range(9)]
        self.NW = NW
        self.NxN = NW * NW
        self.stones = [ Stone(i, Stone.EMPTY) for i in range(self.NxN) ]
        m = NW//2
        n = m - 1
        self.stones[NW*n+n].setColor( Stone.WHITE )
        self.stones[NW*n+m].setColor( Stone.BLACK )
        self.stones[NW*m+n].setColor( Stone.BLACK )
        self.stones[NW*m+m].setColor( Stone.WHITE )
        nrnd = randint(0,1)
        if nrnd==0:
            self.stones[NW * n + n].flipColor()
            self.stones[NW * n + m].flipColor()
            self.stones[NW * m + n].flipColor()
            self.stones[NW * m + m].flipColor()
        self.nBlack = self.nWhite = 2
        self.nEmpty = self.NxN - (self.nBlack + self.nWhite)
        self.state = Board.GAME

    def winner(self):
        self.nBlack = self.nWhite = self.nEmpty = 0
        for i in range(self.NxN):
            if self.stones[i].getColor() == Stone.HUMAN_ID:
                self.nBlack += 1
            elif self.stones[i].getColor() == Stone.MACHINE_ID:
                self.nWhite += 1
            else:
                self.nEmpty += 1
        if self.nBlack+self.nWhite == self.NxN:
            if self.nBlack < self.nWhite:
                return Board.MACHINE_WIN
            elif self.nBlack > self.nWhite:
                return Board.HUMAN_WIN
            else:
                return Board.DRAW
        return Board.GAME

    def print(self):
        print(f"WHITE:{self.nWhite}, BLACK:{self.nBlack}")
        print(' ',end=' ')
        for x in range(self.NW):
            print(f"{x}", end=' ')
        print("→x")
        for y in range(self.NW):
            print(f"{y:}", end=' ')
            for x in range(self.NW):
                print(f"{self.stones[ y*self.NW+x ].getFigure()}", end=' ')
            print()
        print("↓\ny")

    def check_flip(self, stone):
        y = stone.getLocate()//self.NW
        x = stone.getLocate()%self.NW
        self.nKoma[8] = 0
        if self.stones[y*self.NW+x].getColor() == Stone.EMPTY:
            for i1 in range( len(self.nKoma)-1 ):   # 0,1,2,3,4,5,6,7
                m1, m2 = x, y
                self.nKoma[i1] = s = ct = 0
                while s<2:
                    m1 += Board.UNITV[i1][0]
                    m2 += Board.UNITV[i1][1]
                    if (0<=m1<self.NW) and (0<=m2<self.NW):
                        color = self.stones[m2*self.NW+m1].getColor()
                        if color==Stone.EMPTY:
                            s = 3
                        elif color==stone.getColor():
                            s=2 if s==1 else 3
                        else:
                            s=1
                            ct += 1
                    else:
                        s=3
                if s==2:
                    self.nKoma[8] += ct
                    self.nKoma[i1] = ct
        return self.nKoma[8]

    def flip_list(self, stone):
        self.check_flip(stone)
        y = stone.getLocate()//self.NW
        x = stone.getLocate()%self.NW
        flipped = []
        for i1 in range( len(self.nKoma)-1 ):   # 0,1,2,3,4,5,6,7
            m1, m2 = x, y
            for i2 in range( self.nKoma[i1] ):
                m1 += Board.UNITV[i1][0]
                m2 += Board.UNITV[i1][1]
                if (0 <= m1 < self.NW) and (0 <= m2 < self.NW):
                    z = m2*self.NW+m1
                    self.stones[z].setColor( stone.getColor() )
                    flipped.append( z )
        z = y*self.NW+x
        self.stones[z].setColor( stone.getColor() )
        flipped.append(z)
        return flipped

    def empty_list(self):
        list = [v for v in range(self.NxN) if self.is_empty(v)]
        return list

    def is_empty(self, te):
        return True if self.stones[te].getColor()==Stone.EMPTY else False

    def can_put(self, te, color):
        if 0<=te<self.NxN:
            if 0<self.check_flip( Stone(te, color) ):
                return True
        return False

    def puttable(self, te, color):
        if self.can_put(te, color):
            return self.nKoma

    def puttable_list(self, color):
        emptyl = self.empty_list()
        komal = []
        telist = []
        for n in emptyl:
            w = self.puttable(n, color)
            if w:
                telist.append(n)
                komal.append(w)
        return telist, komal

    def put_stone(self, te, color):
        stone = Stone(te, color)
        flist = self.flip_list( stone )
        self.set_stones(flist, color)
        self.stones[te] = stone
        if color==Stone.BLACK:
            self.nBlack = len(flist) + 1
        else:
            self.nWhite = len(flist) + 1
        return flist

    def set_stones(self, flist, color):
        for z in flist:
            self.stones[z] = Stone(z, color)
        self.stones[z] = Stone(z, Stone.EMPTY)

    def reset_stones(self, te, flist, color):
        if color==Stone.WHITE:
            c = Stone.BLACK
        elif color==Stone.BLACK:
            c = Stone.WHITE
        self.stones[te] = Stone(te,c)
        for z in flist:
            self.stones[z] = Stone(z, c)
        self.stones[z] = Stone(z, Stone.EMPTY)

    def debug_print(self):
        for y in range(self.NW):
            for x in range(self.NW):
                n = y*self.NW + x
                print(self.stones[n].color, end=' ')
            print()
        print('\n')
\end{lstlisting}

\subsection{Humanクラス}

人の打つ石の手を決めているのは、このクラスのメソッドTe()です

プロンプトメッセージ('Your turn. [yx] or "pass": ')を表示して、
入力文字列を受け取ります(instr)

入力文字列(instr)の中に文字列'pass'が含まれていたなら、まず、
石を置ける場所があったにもかかわらず、誤ってパスを指示したかもしれない
可能性をチェック(puttable())します

もし、石を置く場所がある局面なら、パスをしてはいけないので、
プロンプトメッセージまで処理を戻します(continue)

石を置く場所がなくてパスの指示をしたのなら、
パスの指示を返します(return Board.PASS)

入力文字列にスロット番号として、行番号yと列番号xが[yx]として指定されたならば、
yxという2桁の数字文字を、それぞれ整数に変換(serial\_num())しています

yとxが数字から整数値に変換できたなら、スロットの位置番号を計算
(slot = cy * board.NW + cx)できます

計算したスロット番号に、その石を置けるかどうかをチェック
(board.can\_put(slot, self.color))して、置けない場合は
最初の文字列入力プロンプトまで処理を戻します

チェックした結果、石を置くことのできるスロット番号だったなら、
反復(while True)を抜けて(break)、そのスロット番号を返します(return slot)

\begin{lstlisting}[caption=Human class,label=othello03]
from Board import Board
from Stone import Stone

class Human:
    def __init__(self, color=Stone.BLACK):
        self.color = color

    def Te(self, board):
        def puttable():
            elist =  board.empty_list()
            for v in elist:
                if board.can_put(v, self.color):
                    return True
            return False

        while True:
            instr = input('Your turn. [yx] or "pass": ')
            if 'pass' in instr:
                if puttable():
                    continue
                return Board.PASS
            if 1<len(instr) and self.numP( instr[0] ) and self.numP( instr[1] ):
                cy = self.serial_num( instr[0] )
                cx = self.serial_num( instr[1] )
                slot = cy * board.NW + cx
                if board.can_put(slot, self.color):
                    break
        print()
        return slot

    def numP(self, xy):
        if '0' <= xy <= '9':
            return True
        return False

    def serial_num(self, xy):
        def fromA(xy):
            return int(xy - 'a')

        if self.numP(xy):
            nxy = int( xy )
        else:
            nxy = fromA(xy)+10
        return nxy
\end{lstlisting}

\subsection{Machineクラス}

コンピュータ(Machine)の打つ手を決めているのが、このクラスのメソッド、Te()です

①まず最初に、board.puttable\_list()によって石を置ける場所の一覧をtelistというリストとして取得します

②打つ手のリストの要素がない場合は、Board.PASSを返します

③打つ手のリストの要素が一つしかない場合は、迷うことなくその手を返します

④'random'戦略が選ばれている場合は、telistの中から乱数で選んだ手を返します

⑤'maxflip'戦略が選ばれている場合は、BoardクラスのnKomaリストの一番最後の要素に、
反転できる石の数の総数が入っているので、最も多く反転できる手を返します

⑥'minimax'戦略が選ばれている場合は、MINIMAX法による手が選ばれますが、
MINIMAX法は、勝敗がつくまで、最後まで対戦を試行するので、盤面のサイズが
最小値(N=4)の場合でも、とても応答に時間がかかります。

⑦'alphabeta'戦略が選ばれている場合は、MINIMAX法を途中で打ち切る手立てを
含んでいるので、多少高速な応答を期待できますが、、、、、

\begin{lstlisting}[caption=Machine class,label=othello02]
from random import randint
from Board import Board
from Stone import Stone
from Strategy import Strategy

class Machine( Strategy ):
    def __init__(self, color=Stone.WHITE, strategy = 'random'):
        self.color = color
        self.strategy = strategy

    def Te(self, board):
        telist, komalist = board.puttable_list( self.color )
        if not telist:
            return Board.PASS
        if len(telist)==1:
            return telist[0]
        if self.strategy == 'random':
            n = randint( 0, len(telist)-1 )
            return telist[n]
        elif self.strategy == 'maxflip':
            n = v = -1
            for i, k in enumerate(komalist):
                if n < k[-1]:
                    n = k[-1]
                    v = telist[i]
            return v
        elif self.strategy == 'minimax':
            print('thinking ...')
            n = self.bestMove(board, telist)
            print('Your turn.')
            return n
        elif self.strategy == 'alphabeta':
            n = self.bestMoveAB(board, telist)
            return n
\end{lstlisting}

\subsection{Strategyクラス}

このクラスは、Machineクラスのスーパークラスとして、
Machineクラスに継承させて使うようにします

ここには、MINIMAX法とAlpha-Beta刈りの2つを記述しています

\begin{lstlisting}[caption=Strategy class,label=othello06]
from Board import Board
from Stone import Stone

class Strategy:
    INFINITY = 100000
    def __init__(self):
        pass

    # MiniMax
    def bestMove(self, board, telist):
        bestEval = -Strategy.INFINITY
        bestMove = -1
        for n, v in enumerate(telist):
            flist = board.put_stone(v, self.color)
            eval = self.minimax(board, 0, False)
            if bestEval<eval:
                bestEval = eval
                bestMove = v
            board.reset_stones(v, flist, self.color)
        return bestMove

    def minimax(self, board, depth, isMaximizingPlayer):
        def evaluate(depth):
            if board.state == Board.MACHINE_WIN:
                board.state = Board.GAME
                return board.NxN-depth     #return NxN
            elif board.state == Board.HUMAN_WIN:
                board.state = Board.GAME
                return depth-board.NxN     #return -NxN
            else:
                self.state = Board.GAME
                return 0

        board.winner()
        if board.state != Board.GAME:
            return evaluate(depth)

        color = Stone.MACHINE_ID if isMaximizingPlayer else Stone.HUMAN_ID
        telist, _ = board.puttable_list(color)
        bestVal = -Strategy.INFINITY if isMaximizingPlayer else +Strategy.INFINITY
        for n, v in enumerate(telist):
            flist = board.put_stone(v, color)
            value = self.minimax(board, depth + 1, not isMaximizingPlayer)
            bestVal = max(value, bestVal) if isMaximizingPlayer else min(value, bestVal)
            board.reset_stones(v, flist, color)
        return bestVal

    # Alpha-Beta
    def bestMoveAB(self, board, telist):
        bestEval = -Strategy.INFINITY
        bestMove = -1
        for n, v in enumerate(telist):
            flist = board.put_stone(v, self.color)
            eval = self.minimaxab(board, 8, 0, False, -Strategy.INFINITY, +Strategy.INFINITY)
            if bestEval<eval:
                bestEval = eval
                bestMove = v
            board.reset_stones(v, flist, self.color)
        return bestMove

    def minimaxab(self, board, node, depth, isMaximizingPlayer, alpha, beta):
        def evaluate(depth):
            if board.state == Board.MACHINE_WIN:
                board.state = Board.GAME
                return board.NxN-depth     #return NxN
            elif board.state == Board.HUMAN_WIN:
                board.state = Board.GAME
                return depth-board.NxN     #return -NxN
            else:
                self.state = Board.GAME
                return 0

        board.winner()
        if board.state != Board.GAME:
            return evaluate(depth)
        elif node==0:
            n = board.nBlack - board.nWhite
            return n if isMaximizingPlayer else -n

        color = Stone.MACHINE_ID if isMaximizingPlayer else Stone.HUMAN_ID
        telist, _ = board.puttable_list(color)
        bestVal = -Strategy.INFINITY if isMaximizingPlayer else +Strategy.INFINITY
        for n, v in enumerate(telist):
            flist = board.put_stone(v, color)
            value = self.minimaxab(board, node-1, depth+1, not isMaximizingPlayer, alpha, beta)
            board.reset_stones(v, flist, color)
            if isMaximizingPlayer:
                bestVal = max(value, bestVal)
                alpha = max(alpha, bestVal)
            else:
                bestVal = min(value, bestVal)
                beta = min(beta, bestVal)
            if beta<=alpha:
                break
        return bestVal
\end{lstlisting}



%\section*{謝辞}
%\addcontentsline{toc}{chapter}{謝辞}
%
\begin{thebibliography}{99}
  \bibitem{1}
\end{thebibliography}
%
% END DOCUMENT
\end{document}
%
